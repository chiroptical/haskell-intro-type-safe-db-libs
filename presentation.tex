%\PassOptionsToPackae{gray}{xcolor}
\documentclass[hyperref={pdfpagelabels=false},12pt]{beamer}
\setbeamertemplate{frametitle}[default][center]
\mode<presentation>
{
 \usetheme{Warsaw}      % or try Darmstadt, Madrid, Warsaw, ...
 \usecolortheme{default} % or try albatross, beaver, crane, ...
 \usefonttheme{default}  % or try serif, structurebold, ...
 \setbeamertemplate{footline}[frame number]
 \setbeamertemplate{caption}[numbered]
}

\usepackage[utf8]{inputenc}
\usepackage{mathtools}
\usepackage{bm}
\usepackage{helvet}
\usepackage{listings}
\usepackage{gensymb}
\usepackage{array}
\usepackage{times}
\usepackage{xcolor}
\usepackage{default}
\usepackage{ulem}
\usepackage{minted}
\usepackage{hyperref}
\usepackage{booktabs}

% Great Commands
\newcommand{\ig}[2]{\includegraphics[width=#1\linewidth]{#2}}
\newcommand{\mybutton}[2]{\hyperlink{#1}{\beamerbutton{{#2}}}}
\newcommand{\myvbutton}[2]{\vfill\hyperlink{#1}{\beamerbutton{{#2}}}}
\newcommand{\code}[2]{\mintinline{#1}{#2}}
\newcommand{\haskell}[1]{\mintinline{haskell}{#1}}
\newcommand{\sql}[1]{\mintinline{sql}{#1}}
\newcommand{\unnamedUrl}[1]{\href{#1}{\color{blue}{#1}}}
\newcommand{\namedUrl}[2]{\href{#1}{\color{blue}{#2}}}
\newcommand{\pygment}[3]{
  \inputminted[frame=single,framesep=2mm,linenos,fontsize=#1]{#2}{#3}
}
\newcommand{\pygmentLines}[5]{
  \inputminted[frame=single,framesep=2mm,linenos,fontsize=#1,firstline=#2,lastline=#3,autogobble]{#4}{#5}
}

% Color Scheme
\definecolor{pittblue}{RGB}{28,41,87}
\definecolor{pittgold}{RGB}{205,184,125}
\setbeamercolor{structure}{fg=pittgold}
\setbeamercolor{button}{bg=pittblue}

\title[CRUD]{{An Introduction to Type-Safe Database Libraries in Haskell}}
\author[CRUD]{{Barry Moore II}}
\institute[CRC]{Center for Research Computing \\ University of Pittsburgh}
\date{}

\beamertemplatenavigationsymbolsempty

\begin{document}

\begin{frame}[plain]
\titlepage
\end{frame}

\begin{frame}{Outline}
  \begin{itemize}
    \item Overview of Database Libraries in Haskell
    \item Modelling the Data
    \item Building Queries
    \item Running Queries
    \item Conclusions
  \end{itemize}
\end{frame}

\begin{frame}{Disclaimers}
  \begin{itemize}
    \item I don't have a lot of experience with databases
    \item Experience: MongoDB with PyMongo and Python Dataset with SQLite
    \item The databases I ``maintain'' could probably be simple files
    \item Motivation for this talk: \unnamedUrl{https://williamyaoh.com/posts/2019-12-14-typesafe-db-libraries.html}
    \item Code is available: \unnamedUrl{https://github.com/barrymoo/haskell-intro-type-safe-db-libs}
  \end{itemize}
\end{frame}

\begin{frame}{Database Libraries in Haskell}
  \begin{itemize}
    \item Embed SQL in Haskell (the ``simple'' packages)
    \begin{itemize}
      \item SQLite: \unnamedUrl{https://hackage.haskell.org/package/sqlite-simple}
      \item PostgreSQL: \unnamedUrl{https://hackage.haskell.org/package/postgresql-simple}
      \item MySQL: \unnamedUrl{https://hackage.haskell.org/package/mysql-simple}
    \end{itemize}
    \item What eDSLs are available for writing type-safe database queries?
    \begin{itemize}
      \item Persistent+Esqueleto: Backend agnostic, heavy use of TH
      \item Beam: Backend agnostic, no TH
      \item Opaleye: PostgreSQL specific, happy medium between the above
    \end{itemize}
  \end{itemize}
\end{frame}

\begin{frame}{What would we like to store in our database?}
  \pygment{\tiny}{haskell}{code/Person.hs}
\end{frame}

\begin{frame}{Persistent+Esqueleto}
  \centering \Huge Persistent+Esqueleto
\end{frame}

\begin{frame}{Persistent+Esqueleto: Type Definitions}
  \pygmentLines{\scriptsize}{18}{18}{haskell}{code/pers-esql/src/Person.hs}
  \pygmentLines{\scriptsize}{20}{37}{haskell}{code/pers-esql/src/Person.hs}
\end{frame}

\begin{frame}{Persistent+Esqueleto: Creating Initial Table}
  \pygmentLines{\scriptsize}{48}{49}{haskell}{code/pers-esql/src/Person.hs}
  \begin{center}
    Approximate SQL
  \end{center}
  \pygment{\scriptsize}{sql}{code/sql/createTable.sql}
\end{frame}

\begin{frame}{Persistent+Esqueleto: Create/Insert}
  \pygmentLines{\scriptsize}{51}{57}{haskell}{code/pers-esql/src/Person.hs}
  \pygment{\scriptsize}{sql}{code/sql/insertInto.sql}
\end{frame}

\begin{frame}{Persistent+Esqueleto: Read/Get}
  \pygmentLines{\scriptsize}{15}{15}{haskell}{code/pers-esql/src/Person.hs}
  \pygmentLines{\scriptsize}{59}{67}{haskell}{code/pers-esql/src/Person.hs}
  \pygment{\scriptsize}{sql}{code/sql/selectWhere.sql}
\end{frame}

\begin{frame}{Persistent+Esqueleto: Update}
  \pygmentLines{\scriptsize}{69}{77}{haskell}{code/pers-esql/src/Person.hs}
  \pygment{\scriptsize}{sql}{code/sql/updateSetWhere.sql}
\end{frame}

\begin{frame}{Persistent+Esqueleto: Remove/Delete}
  \pygmentLines{\scriptsize}{79}{87}{haskell}{code/pers-esql/src/Person.hs}
  \pygment{\scriptsize}{sql}{code/sql/deleteWhere.sql}
\end{frame}

\begin{frame}{Persistent+Esqueleto: Running a Query}
  \pygmentLines{\scriptsize}{28}{31}{haskell}{code/pers-esql/test/Spec.hs}
  \pygmentLines{\scriptsize}{40}{45}{haskell}{code/pers-esql/test/Spec.hs}
\end{frame}

\begin{frame}{Persistent+Esqueleto: Takeaways}
  \begin{itemize}
    \item I don't like defining the datatypes inside Template Haskell, because
    \begin{itemize}
      \item Implicit data constructor (e.g.
        \haskell{username -> usernameUsername})
      \item The capitalization of fields matters (uniqueness)
    \end{itemize}
    \item It was easy to use and quick to get started
  \end{itemize}
\end{frame}

\begin{frame}{Beam}
  \centering \Huge Beam
\end{frame}

\begin{frame}{Beam: Type Definitions}
  \pygmentLines{\scriptsize}{78}{87}{haskell}{code/beam/src/Person.hs}
\end{frame}

\begin{frame}{Beam: Wrapping and Unwrapping}
  \pygmentLines{\scriptsize}{32}{53}{haskell}{code/beam/src/Person.hs}
\end{frame}

\begin{frame}{Beam: Initial DB Setup}
  \pygmentLines{\scriptsize}{104}{118}{haskell}{code/beam/src/Person.hs}
\end{frame}

\begin{frame}{Beam: Initial DB Setup Cont.}
  \pygmentLines{\scriptsize}{120}{135}{haskell}{code/beam/src/Person.hs}
\end{frame}

\begin{frame}{Beam: Create}
  \pygmentLines{\scriptsize}{146}{148}{haskell}{code/beam/src/Person.hs}
  \pygment{\scriptsize}{sql}{code/sql/insertInto.sql}
\end{frame}

\begin{frame}{Beam: Read/Get}
  \pygmentLines{\scriptsize}{137}{144}{haskell}{code/beam/src/Person.hs}
  \pygment{\scriptsize}{sql}{code/sql/selectWhere.sql}
\end{frame}

\begin{frame}{Beam: Update}
  \pygmentLines{\scriptsize}{150}{159}{haskell}{code/beam/src/Person.hs}
  \pygment{\scriptsize}{sql}{code/sql/updateSetWhereEmail.sql}
\end{frame}

\begin{frame}{Beam: Remove/Delete}
  \pygmentLines{\scriptsize}{161}{168}{haskell}{code/beam/src/Person.hs}
  \pygment{\scriptsize}{sql}{code/sql/deleteWhere.sql}
\end{frame}

\begin{frame}{Beam: Running a Query}
  \pygmentLines{\scriptsize}{37}{42}{haskell}{code/beam/test/Spec.hs}
\end{frame}

\begin{frame}{Beam: Takeaways}
  \begin{itemize}
    \item Hint: use partial type signatures to ease type definitions
      \pygment{\scriptsize}{haskell}{code/partialTypeSigs.hs}
    \begin{itemize}
      \item \unnamedUrl{https://github.com/barrymoo/slurm-proposals/blob/master/src/Queries.hs\#L31}
    \end{itemize}
    \item Migrations are tricky, probably avoid them at the beginning
    \item Boilerplate gets annoying quickly
  \end{itemize}
\end{frame}

\begin{frame}{Opaleye}
  \centering \Huge Opaleye
\end{frame}

\begin{frame}{Opaleye: Type Definitions}
  \pygmentLines{\scriptsize}{42}{62}{haskell}{code/opaleye-example/src/Person.hs}
\end{frame}

\begin{frame}{Opaleye: Table Definition}
  \pygmentLines{\scriptsize}{66}{86}{haskell}{code/opaleye-example/src/Person.hs}
\end{frame}

\begin{frame}{Opaleye: Create/Insert}
  \pygmentLines{\scriptsize}{98}{111}{haskell}{code/opaleye-example/src/Person.hs}
  \pygment{\scriptsize}{sql}{code/sql/insertInto.sql}
\end{frame}

\begin{frame}{Opaleye: Read/Get}
  \pygmentLines{\scriptsize}{127}{133}{haskell}{code/opaleye-example/src/Person.hs}
  \pygment{\scriptsize}{sql}{code/sql/selectWhere.sql}
\end{frame}

\begin{frame}{Opaleye: Update}
  \pygmentLines{\scriptsize}{113}{125}{haskell}{code/opaleye-example/src/Person.hs}
  \pygment{\scriptsize}{sql}{code/sql/updateSetWhereEmailAge.sql}
\end{frame}

\begin{frame}{Opaleye: Remove/Delete}
  \pygmentLines{\scriptsize}{89}{96}{haskell}{code/opaleye-example/src/Person.hs}
  \pygment{\scriptsize}{sql}{code/sql/deleteWhere.sql}
\end{frame}

\begin{frame}{Opaleye: Running a Query}
  \pygmentLines{\scriptsize}{26}{37}{haskell}{code/opaleye-example/test/Spec.hs}
\end{frame}

\begin{frame}{Opaleye: Takeaways}
  \begin{itemize}
    \item Slightly strange table type that is polymorphic in all fields
    \item No migrations, handled on your own
    \item Arrows could have been a rabbit hole, but docs made them easy enough
    \item Only PostgreSQL backend
    \item Despite this, I enjoyed using Opaleye
  \end{itemize}
\end{frame}

\begin{frame}{Conclusions}
  \begin{itemize}
    \item One library wasn't particularly harder than another to use
    \item ``Time to first query'' is fast with Persistent
    \item Learning curve is more shallow with Beam and Opaleye
    \item Queries in Opaleye are different than Beam/Persistent
    \item I would probably use Opaleye for a new project if it supported SQLite
      backend
    \item So, what will I choose for my next project?
  \end{itemize}
\end{frame}

\begin{frame}{Questions?}
  \centering \Huge Any Questions?
\end{frame}

\end{document}
